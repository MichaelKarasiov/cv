% !TEX TS-program = lualatex

\documentclass[a4paper,11pt]{article}

\usepackage[usenames,dvipsnames]{xcolor}
\usepackage{libertine}
\usepackage{fontawesome}
\usepackage{longtable}
\usepackage[cm]{fullpage}
\usepackage{amssymb}

% headers and footers
\usepackage{fancyhdr}
\usepackage{lastpage}
\pagestyle{fancy}
\fancyfoot[L]{Igal Tabachnik}
\fancyfoot[C]{Page \thepage\ of \pageref*{LastPage}}
\fancyfoot[R]{\today}
\renewcommand{\footrulewidth}{0.4pt}% default is 0pt
\fancyhead{}
\renewcommand{\headrulewidth}{0pt}

% StackOverflow-like tags
% https://tex.stackexchange.com/a/311949/142692
% https://tex.stackexchange.com/questions/387499/how-to-create-a-border-that-looks-like-a-tag
\usepackage{tikz}
\definecolor{tagbg}{RGB}{225,236,244}
\definecolor{tagtxt}{RGB}{88,115,159}
\newcommand{\sotag}[1]{\tikz[baseline]{\node[anchor=base, rounded corners=0.5ex, text height=1.5ex, text depth=.25ex, fill=tagbg, draw=tagbg, text=tagtxt] {#1};}}

% Helpers for adding job entries
\newcommand{\job}[2]{\large\sffamily \textbf{#1} at \textbf{#2}}
\newcommand{\sep}{\multicolumn{2}{c}{}\\}

% tweak the url colors
\usepackage{hyperref}
\definecolor{linkcolor}{rgb}{0,0.2,0.6}
\hypersetup{colorlinks,breaklinks,urlcolor=linkcolor, linkcolor=linkcolor}

% nicer-looking section titles
\usepackage{titlesec}
\titleformat{\section}{\Large\scshape\raggedright}{}{0em}{}[\titlerule]
\titlespacing{\section}{0pt}{1em}{3pt}

\begin{document}

% --------------------TITLE-------------
\par{\centering
		{\Huge \textsc{Igal Tabachnik}
	}\bigskip\par}

\hrule
\vspace{0.5em}
\begin{tabular}{rl}
  \textsc{Phone:}     & +972 54 4766343\\
  \textsc{Email:}     & \href{mailto:hmemcpy@gmail.com}{hmemcpy@gmail.com}\\
  \textsc{Socials:}   & \faHome{} \href{https://hmemcpy.com}{hmemcpy.com} 
                      | $\mathbb{X}$ \href{https://twitter.com/hmemcpy}{@hmemcpy}
                      | \faGithub{} \href{https://github.com/hmemcpy}{github}
                      | \faLinkedin{} \href{https://www.linkedin.com/in/igaltabachnik/}{linkedin}
                      | \faStackOverflow{} \href{https://stackoverflow.com/users/8205/igal-tabachnik}{stackoverflow}
\end{tabular}

\section{Summary}
\begin{tabular}{p{0.9\textwidth}}
  I'm a software developer based in Israel with over 20 years of professional experience across multiple domains and technologies.\\\\

  I am currently working primarily with Scala using the ZIO stack (typesafe, purely functional library for concurrent and asynchronous programming in Scala). I specialize in backend development: implementing highly concurrent and robust services, domain modeling, and testing. I am particularly interested in Developer Experience (DX) and have created compiler and IDE plugins that significantly improve developer productivity.\\\\

  I am an open source contributor and an occasional organizer and guest speaker at conferences and local meetups. I love teaching and helping others to become ``unstuck''.
\end{tabular}

\section{Work Experience}
\begin{longtable}{r|p{0.72\textwidth}}

  \textsc{Apr 2020--Mar 2024} & \job{Founding Engineer and Tech Lead}{Unit}, Israel \\(4 years)
    &\sotag{scala} \sotag{zio} \sotag{infrastructure} \sotag{onboarding} \sotag{developer-experience} \\&\\
    &First employee at Unit. Held several positions across numerous teams.\\&\\
    &Created and maintained most of the backend infrastructure using pure functional programming in Scala, based on the ZIO framework and associated libraries.\\&\\
    &Implemented services and protocols according to their exact specification, using advanced data modeling techniques allowed by the strong type system of Scala.\\&\\
    &Was the go-to person for supporting and onboarding newcomers, troubleshooting, creating educational materials and hands-on workshops. Created internal tools (compiler and IDE plugins) to improve developer productivity. Defined engineering guidelines and testing specifications.\\\sep
  
  \hline
  \multicolumn{2}{r}{\footnotesize\itshape (abbreviated work history below, see the LinkedIn profile for full details)}\\\sep

  \textsc{Jun 2016--Aug 2019} & \job{Senior Software Engineer}{Wix.com}, Israel \\(3 years, 3 months)
    &\sotag{scala} \sotag{functional-programming} \sotag{bazel} \sotag{intellij-plugins}\\&\\
    &Worked in the build infrastructure team, supporting the migration to the \textit{Bazel} build system. Main responsibilities included adding Scala support and contributing fixes to the \textit{Bazel IntelliJ plugin} (maintained by Google), as well as creating internal tools for Wix-specific functionality.\\\sep
  
  \textsc{Jul 2015--May 2016} & \job{Software Developer}{Particular Software}, Israel \\(11 months)
    &\sotag{c\#} \sotag{nservicebus}\\&\\
    &Building NServiceBus and the Particular Platform products.\\\sep
  
  \textsc{Jul 2012--Jan 2015} & \job{Lead Developer}{OzCode} (a CodeValue company), Israel \\(2 years, 7 months)
    &\sotag{c\#} \sotag{roslyn} \sotag{debugging-api} \sotag{visual-studio-extensions}\\&\\
    &OzCode is a Visual Studio extension for debugging productivity. I was responsible for the entire stack, from core product development to prioritizing features and issues, research, as well as marketing and content creation, branding, and artwork direction.\\\sep
  
  \textsc{Mar 2010--Nov 2011} & \job{Senior Software Developer}{Typemock}, Israel \\(1 year, 9 months)
    &\sotag{c\#} \sotag{.net-internals} \sotag{il-weaving} \sotag{aop} \sotag{api-design} \sotag{code-generation}\\&\\
    &Developed a unit testing suite of products for software developers. Mainly responsible for the isolation (mocking) framework, based on the unmanaged \emph{CLR Profiling API} to perform runtime inspection and IL weaving to allow runtime code modifications.\\\sep

  \textsc{Oct 2008--Mar 2010} & \job{Software Developer}{Eternix}, Israel \\(1 year, 6 months)
    &\sotag{c\#} \sotag{webdav} \sotag{winforms} \sotag{unit-testing} \sotag{tdd}\\&\\
    &Lead developer of a WebDAV based file server, implementing and maintaining features such as file encryption, versioning, quota, and user management.\\\sep

  \textsc{Oct 2007--Oct 2008} & \job{Software Developer}{InfoGin}, Israel \\(1 year, 1 month)
    &\sotag{c\#} \sotag{asp.net} \sotag{mobile-web} \sotag{wap}\\&\\
    &Developer in the professional services team, created web applications for mobile devices, based on the specifications provided by customers.\\\sep

  \textsc{Jun 2005--Sep 2007} & \job{Software Developer}{PrizmaSoft}, Israel \\(2 years, 4 months)
    &\sotag{c\#} \sotag{winforms} \sotag{continuous-integration}\\&\\
    &Developed client applications for a business process management system, maintained build and deployment scripts.\\\sep
\end{longtable}

\section{Programming Proficiency}
\begin{tabular}{rl}
  \textsc{Languages:}& Scala (expert), C\#, Java (proficient), Haskell (beginner), F\#, OCaml (familiar), Rust (learning)\\
  \textsc{Patterns:}& ES/CQRS, DDD, OOP, TDD, Property-based testing\\
  \textsc{Technologies:}& ZIO and Typelevel libraries, Postgres, Docker, Kubernetes, Kafka, TestContainers\\
  \textsc{Tools:}& Scala macros, sbt, Pants/Bazel, IntelliJ plugins\\
\end{tabular}

\section{Skills and Accomplishments}
\begin{tabular}{rl}
  \textsc{Languages:}& Hebrew, English, Russian\\
  \textsc{Publications:}& Category Theory for Programmers by Bartosz Milewski (\href{https://github.com/hmemcpy/milewski-ctfp-pdf}{PDF}, \href{https://www.blurb.com/b/9621951-category-theory-for-programmers-new-edition-hardco}{hardcover book})\\
  \textsc{Open-source:}&Contributor to various open source projects (ZIO, various plugins for IntelliJ IDEA)
\end{tabular}

\section{Speaking}
\begin{tabular}{rl}
  \textsc{Video:}&\href{https://www.youtube.com/watch?v=g1EvM4CbUvM}{Journey to Functional Programming} (Wix Engineering, 2017)\\
  &\href{https://www.youtube.com/watch?v=N6ZJwnvTjLA}{Zero to FP (Hebrew)} (Underscore meetup, 2018)\\
  &\href{https://www.youtube.com/watch?v=qBmZJwmd0CA}{Everything you (didn't) want to know about implicits (Hebrew)} (Unit Engineering, 2022)\\
  &\href{https://www.youtube.com/watch?v=eILoMm9t4rI}{The Business of Scala} (Scala Matters Meet-up @ HiBob, 2023)\\  
\end{tabular}

\end{document}
